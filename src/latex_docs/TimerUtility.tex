%%\documentclass[10pt]{article}
\documentclass[fleqn,10pt,letterpaper]{article}
\usepackage{amsmath}
\usepackage{appendix}
\usepackage{amsmath}
\usepackage{fancyvrb}

\usepackage{listings} %include code in your document
\usepackage[section]{placeins}

\usepackage{listings}
\lstloadlanguages{Matlab}
\lstnewenvironment{PseudoCode}[2][]{ \vspace{1ex}\noindent
  \lstset{ 
    language=Matlab,
    basicstyle=\small\ttfamily,
    numbers=left,
     numberstyle=\tiny\color{gray},
    backgroundcolor=\color{white},
    firstnumber=auto,
    stepnumber=1,
    backgroundcolor=\color{white},
    tabsize=4,
    showspaces=false,
    showstringspaces=false,
    numbers=left,
    xleftmargin=.1\textwidth,
    keywords={break,case,catch,continue,else,elseif,end,for,function,
        global,if,otherwise,persistent,return,switch,try,while},
    keywordstyle=\color{blue},
    commentstyle=\color{red},
    stringstyle=\color{dkgreen}
  } 
}{ }


\setlength{\mathindent}{24pt}

\setlength{\textwidth}{6.5in}
\setlength{\oddsidemargin}{0.0in}
\setlength{\evensidemargin}{0.0in}
\setlength{\textheight}{9.0in}
\setlength{\topmargin}{0.0in}
\setlength{\headheight}{0.0in}
\setlength{\headsep}{0.0in}
\setlength{\footskip}{0.5in}

\usepackage{setspace}\onehalfspacing
\AtBeginDocument{%
  \addtolength\abovedisplayskip{-0.5\baselineskip}%
  \addtolength\belowdisplayskip{-0.5\baselineskip}%
  %%\addtolength\abovedisplayshortskip{-0.5\baselineskip}%
  %%\addtolength\belowdisplayshortskip{-0.5\baselineskip}%
}

\usepackage{hyperref}
\hypersetup{
    colorlinks,
    citecolor=green,
    filecolor=cyan,
    linkcolor=red,
    urlcolor=magenta
}

\newcommand\abs[1]{\left|#1\right|}
\newcommand{\overbar}[1]{\mkern 1.5mu\overline{\mkern-1.5mu#1\mkern-1.5mu}\mkern 1.5mu}



%%---------------------------------------------------------------------------%%
%% DEFINE SPECIFIC ENVIRONMENTS HERE
%%---------------------------------------------------------------------------%%
\newenvironment{codelisting}
{\begin{list}{}{\setlength{\leftmargin}{2em}}\item\scriptsize\normalsize}
{\end{list}}


\usepackage{hyperref}
%\usepackage{algorithm}
%\usepackage{algpseudocode}


%% Add tikz package for creating some of the figures
\usepackage{tikz}
\usetikzlibrary{shapes,arrows}
% Define block styles
\tikzstyle{decision} = [diamond, draw, fill=blue!20, 
    text width=4.5em, text badly centered, node distance=3cm, inner sep=0pt]
\tikzstyle{block} = [rectangle, draw, fill=blue!20, 
    minimum width=6em, text centered, rounded corners, minimum height=4em]
\tikzstyle{line} = [draw, -latex']
\tikzstyle{cloud} = [draw, ellipse,fill=red!20, node distance=3cm,
    minimum height=2em]
    


\begin{document}

\title{Timer Utility}
\author{Mark Berrill}
\maketitle



%%---------------------------------------------------------------------------%%
%% Basics
%%---------------------------------------------------------------------------%%
\section{Introduction}

This program is a utility program to help time routines in a user's application.
More specifically, it fulfills an need where users need to understand and monitor 
the performance of applications in a production environment.  It provides the 
capability between a user implementing specific calls to get the elapsed time 
(such as MPI\_Wtime), and a detailed profiler such as gprof or VampirTrace.  
Using individual calls to get the time and recording the time elapsed requires
excessive effort on the user's part including tracing the total time spend in 
different routines and presenting it in an easy to understand manor, while many
automatic profilers can be heavy weight, provide an excess of information that
has to be interpreted, and can affect performance which prevent's its use in a 
production environment.  Additionally the capabilities provided by this program
are thread-safe and MPI aware, and can be used in a hybrid and high performance
environment.  The timers have been used to profile applications with ~100,000
processes.  Some libraries may provide similar functionality at part of a larger 
infrastructure, and it is not the goal of this document to compare features, but 
this program may be used to provide additional capabilities or used without requiring
incorporating a large library.  
Installation of the program is described in section \ref{sec:install}.  
Simple usage is covered in section \ref{sec:usage}.  
Analysis of the timer files is covered in section \ref{sec:analysis},
while the graphical user interface is covered in section \ref{sec:gui}.
Advanced topics are covered in section \ref{sec:advanced}.  



%%---------------------------------------------------------------------------%%
%% Installation
%%---------------------------------------------------------------------------%%
\section{Installation}  \label{sec:install}

This program is designed to be easily incorporated into a user's application.  
There are two methods of linking the timers into a project.  The first method 
is by using the timer utility as a third party library that is installed, and 
then linked to the user's application like any other third party library.  
The installation of this application and installation of the library is covered
in sub-section \ref{sec:install_library}.  The second method is to incorporate 
the source code directly into the user's application.  The timer utility consists
of a handful of files that can be found in the src directory.  They are 
"ProfilerApp.h", "ProfilerApp.cpp", "ProfilerAppMacros.h", and "ProfilerAppMacros\_C.h".
The files can be copied to the user's application and built as part of their 
build process.  The major disadvantage to the later method is that changes or
improvements to the timer utility requires replacing the files in the user's 
source tree.  Installation of the tests and the documentation must be done by 
installing the library as described in section \ref{sec:install_library}. 

\subsection{Compiling}  \label{sec:install_library}

Installing the timer utility library requires a C++ compiler and CMake.  
We assume that the user is familiar with compiling software on their machine
and will be using the same compiler as they use to build their application.
Once the application is downloaded, the user needs to choose a build directory
and an install directory.  The build and install directories may be the same
path, but must be different from the source tree.  To configure, the user should
create a CMake script.  An example is given below:
\begin{Verbatim}[baselinestretch=1.0]
   cmake                                                \
      -G "Unix Makefiles"                               \
      -D TIMER_INSTALL_DIR:PATH="path to install"       \
      -D CMAKE_C_COMPILER:PATH=mpicc                    \
         -D CFLAGS="-fPIC"                              \
      -D CMAKE_CXX_COMPILER:PATH=mpic++                 \
         -D CXXFLAGS="-fPIC"                            \
      -D CMAKE_BUILD_TYPE=Debug                         \
      -D USE_MATLAB=0                                   \
         -D MATLAB_DIRECTORY:PATH=/usr/local/MATLAB/R2014a \
      "path to src"
\end{Verbatim}
The first argument is the generator that CMake will create and available generators
can be found using "cmake --help".  TIMER\_INSTALL\_DIR is an optional argument
that controls the install path of the libraries.  The default path is the build directory.
The next arguments are the C and C++ compilers
to use and any optional compile flags that are required.  CMAKE\_BUILD\_TYPE 
indicates the desired build type (Debug, Release).  If MATLAB is installed,
this flag can be set and the directory specified.  This is necessary to install the 
graphical user interface that is described in section \ref{sec:gui} or the 
MATLAB interface.  The final argument is the path to the source tree.
Example configure scripts can be found in the build folder.

Once the configure script has been generated, the user should cd to build directory
and run the configure script.  This will generate build files for the platform.
Assuming the user is generating "Unix Makefiles", the user then simply types 
"make install" to compile the code and install the files.  The documentation
can be compiled using "make docs" and will be installed in the doc folder.  

Under normal usage it is possible most commands can utilized a set of macro 
functions.  If the library is not used, the user can leave the timer commands in
their source code by creating dummy headers that replace these macros.
See \ref{sec:advanced} for advanced cases that will not fit into this model.
Examples of these headers are given below: 
\begin{Verbatim}[baselinestretch=1.0]
ProfilerApp.h:
    #define PROFILE_START(...)       do {} while(0)
    #define PROFILE_STOP(...)        do {} while(0)
    #define PROFILE_START2(...)      do {} while(0)
    #define PROFILE_STOP2(...)       do {} while(0)
    #define PROFILE_SCOPED(...)      do {} while(0)
    #define PROFILE_SYNCRONIZE()     do {} while(0)
    #define PROFILE_SAVE(...)        do {} while(0)
    #define PROFILE_STORE_TRACE(X)   do {} while(0)
    #define PROFILE_ENABLE(...)      do {} while(0)
    #define PROFILE_DISABLE()        do {} while(0)
    #define PROFILE_ENABLE_TRACE()   do {} while(0)
    #define PROFILE_DISABLE_TRACE()  do {} while(0)
    #define PROFILE_ENABLE_MEMORY()  do {} while(0)
    #define PROFILE_DISABLE_MEMORY() do {} while(0)
\end{Verbatim}



%%---------------------------------------------------------------------------%%
%% Usage
%%---------------------------------------------------------------------------%%
\section{Usage}  \label{sec:usage}

Using the timer utility is relatively simple, consisting of inserting a few macro
commands into the user's application.  These commands allow the user to easily create
timers that will record the minimum/maximum/total time spent in the timer and the
number of calls to the timer.  In addition, information about which timers are 
active when another timer is called is also computed/stored.  This information
can be used to determine where a given timer's time is spent.  The timer data can be 
analyzed using the timer files directly (see section \ref{sec:analysis}) or using 
the provided GUI (see section \ref{sec:gui} to visualize both the timer data and 
time spent in sub-timers.  
The available commands are: \\

\hangindent=2em\noindent
PROFILE\_START(name,level):  This will start a timer.  The arguments are the
    timer name and an optional level.  The timer is identified by the name
    of the timer and the file.  Multiple timers with the same name in the same 
    file are not allowed.  The filename and line numbers are automatically added.
    Error checking to ensure that there is only one timer with a given name per
    file is automatically performed.  Additional check to ensure the timer is not
    currently running are also performed.  If either of these tests failed, the 
    timer utility with throw a std::exception with the given error.  
    Sometimes it is necessary to start a timer on different lines, for example
    starting within a case statement or starting/stopping a timer multiple times
    in a single function.  To allow this there is a second macro PROFILE\_START2
    that take identical argments, but does not check the line number.  The user should
    use PROFILE\_START for the primary call, and PROFILE\_START2 for all remaining calls.
    The level number is optional with values from 0-127.  If specified it indicates 
    the level of the timer, with all levels higher than the runtime level automatically
    disabled.  This allows the user to enable/disable the level of detail for their 
    profiling at runtime (see PROFILE\_ENABLE).  The default level if level is not specified
    is 0.  

\hangindent=2em\noindent
PROFILE\_STOP(name,level):  This will start a timer.  It contains the same arguments
    as PROFILE\_START and will exhibit identical behavior except that it stops a
    timer instead of starting one.

\hangindent=2em\noindent
PROFILE\_SCOPED(object,name,level):  This creates a helper object to start/stop a timer.
    More specifically, it will create a timer object that will start a timer with the given
    name and level like PROFILE\_START.  The timer will automatically stop when the object
    looses scope.  This can be helpful when a routine has several return statements.  
    Unlike PROFILE\_START / PROFILE\_STOP it is recursion safe, automatically appending "-1",
    "-2", ... to the timer name to indicate the recursion level.  There is a slightly higher 
    cost to using PROFILE\_SCOPED due to the cost of creating/destroying a C++ class object.
    Like PROFILE\_START / PROFILE\_STOP the level is optional.  

\hangindent=2em\noindent
PROFILE\_SAVE(filename,global):  This will save the current timer data to a file (or set 
    of files) with the name given by filename.  The program will automatically append the
    appropriate suffix to the filename (e.g. ".timer").  The global argument is an optional
    argument specifying if we want to write a single global timer file or one file/rank (default).
    Note that if a single file is written, it will require all processors to call this macro
    and will require communication on MPI\_COMM\_WORLD.

\hangindent=2em\noindent
PROFILE\_ENABLE(level):  This enables the timers.  This should be called at the begining of
    the application.  The argument level is optional (default value is 0), and indicates the
    level of timers that will be enabled.  All timers with a value greater than this value
    are automatically disabled.

\hangindent=2em\noindent
PROFILE\_DISABLE():  This disables the timers and deletes all data.  This allows the user to 
    reset the timer data.

\hangindent=2em\noindent
PROFILE\_ENABLE\_TRACE():  This enables trace level timer data.  The trace data records the
    time at which each call to start/stop a timer is called.  This can be utilized to generate
    a trace graph with the time at which each timer is active as a function to the wall time.
    See section \ref{sec:gui} for documentation on the GUI that can generate this graph.  

\hangindent=2em\noindent
PROFILE\_DISABLE\_TRACE():  This disables the trace level data.

\hangindent=2em\noindent
PROFILE\_ENABLE\_MEMORY():  This enables memory statistics.  When enabled, this will record 
    the memory in use by the application with each call to start/stop.  This can be utilized
    to generate a plot of the memory used as a function of time in the application.  This is
    useful to help locate memory problems and can be combined with trace data and visualized with
    the GUI.  

\hangindent=2em\noindent
PROFILE\_DISABLE\_MEMORY():  This disables the memory statistics.  

\hangindent=2em\noindent
PROFILE\_PROFILE\_SYNCHRONIZE():  When start/stop is called, the time relative to the start
    of the program is recorded.  This is then used when trace or memory level data is used.
    When using MPI, the different processes may actually start at different times due to 
    delays starting each process.  The user can call this function at the start of the 
    application to synchronize the times between the processes.  This will ensure the times 
    recorded match for all processes.  This required global communication.



%%---------------------------------------------------------------------------%%
%% Usage
%%---------------------------------------------------------------------------%%
\section{Timer Files}  \label{sec:analysis}

The timer utility writes a set of files depending on the timer properties set.  
The program may write 1 file for the timers or 1 file/rank depending on the 
options given to PROFILE\_SAVE.  In addition there are 3 types of files that 
may be written.  The first is the timer files (.timer) which is an ASCII file
that is human readable.  This file contains the timer summaries.  The other file
types are the trace (.trace) and memory (memory) files.  These are binary files 
that can be read by load function in the timer class (see section \ref{sec:advanced}).  
The format of the timer files consists of two parts.  The first section is
a summary for the file that contains a list of all timers and their results. 
This is the section that is meant to be human readable, and can be opened by any
editor.  The second section contains details about the timers that are read by 
the GUI (\ref{sec:gui}).  These are generally not useful for direct interpretation.
The format for the first section is:

\begin{Verbatim}[fontsize=\footnotesize,baselinestretch=1.0]
        Message              Filename         Thread  Start Line  Stop Line  N_calls  Min Time  Max Time  Total Time
--------------------------------------------------------------------------------------------------------------------
                MAIN       run_ray_trace.cpp     0        74        675         1     1440.894  1440.894    1440.894
           Main Loop       run_ray_trace.cpp     0       353        660         1     1439.656  1439.656    1439.656
               apply            RayTrace.cpp     0      2395       2714       948        0.703     1.979    1428.089
      Calculate gain            RayTrace.cpp     0      2475       2581       948        0.699     0.867     723.456
  update_populations   update_ionization.cpp     0       150        490       948        0.696     0.864     720.733
       Calculate ASE            RayTrace.cpp     0      2584       2610       948        0.000     1.123     704.450
       propagate_ASE            RayTrace.cpp     1       530        587       868        0.377     0.958     650.758
       propagate_ASE            RayTrace.cpp     2       530        587       868        0.408     0.979     660.674
       propagate_ASE            RayTrace.cpp     3       530        587       868        0.334     0.952     650.014
       propagate_ASE            RayTrace.cpp     4       530        587       868        0.335     0.994     635.733
 advance_populations   update_ionization.cpp     1       712        743      3606        0.085     0.388     625.076
 advance_populations   update_ionization.cpp     2       712        743      3362        0.093     0.430     624.464
 advance_populations   update_ionization.cpp     3       712        743      3378        0.092     0.446     622.196
 advance_populations   update_ionization.cpp     4       712        743      3212        0.087     0.435     615.354
\end{Verbatim}

The first two columns contain the timer name and filename for the timer.  
The thread column indicates which threads contained the thread that called the timer.
The next two columns show the start and stop lines corresponding to the lines that
contain the PROFILE\_START and PROFILE\_STOP commands.  A value of -1 indicates that
there is no such line.  This will happen if PROFILE\_START2 or PROFILE\_STOP2 is used,
or if PROFILE\_SCOPED is used, there is no value for the stop line.  
The final columns indicate the number of calls to the timer, the minimum time for a
single call, the maximum call for a call, and the total time spent in the timer.  
If a global summary is written, then these values are averaged across all ranks.  
More detailed analysis including the load balance, the sub-timers, inclusive/exclusive
time, etc. can be obtained by using the graphical user interface (section \ref{sec:gui}).  



%%---------------------------------------------------------------------------%%
%% Usage
%%---------------------------------------------------------------------------%%
\section{Graphical User Interface}  \label{sec:gui}

The graphical user interface provides a way to utilize the additional information
that the timer utility provides for a more complete analysis.  This interface 
is currently written in MATLAB.  To install a gui, first make sure that MATLAB
is set up to compile C++ files using the mex interface.  To do this, use the command
"mex -setup" within MATLAB.  Note that this is only required once.  Then configure 
the application with "-D USE\_MATLAB=1" and the MATLAB directory specified to install 
the gui directory (will be located in the install/gui folder).  

Once the gui has been installed, open MATLAB (requires the MATLAB display) and 
change to the gui directory and type "load\_timer".  This will open a MATLAB gui.  
Load a timer set using the "Load" button, navigating to the appropriate timer 
file and open it.  The first time that this is done, there may be a few messages 
that indicate that MATLAB is compiling some files (this is normal).  
Once the file has been loaded the timer list will be populated.  If the timer file
is part of a set (e.g. a parallel program with 1 file/rank), the data from all files 
will automatically be loaded.  



%%---------------------------------------------------------------------------%%
%% Usage
%%---------------------------------------------------------------------------%%
\section{Advanced Topics}  \label{sec:advanced}

In section \ref{sec:usage} the macro interface was described.  This is a simplified
interface that can be used for most use cases.  However the timer utility consists
of a C++ class (ProfilerApp) that has a range of capabilities.  This includes the
ability to create multiple profiler objects within an application.  The macro interface
wraps the function calls to this class on a global object for convince.  All of 
the functions available are described through the doxygen documentation and the global
object containing the default object is "global\_profiler".  Some specific capabilities 
are discussed below.

\subsection{Loading Timer Results}

Users can load timer files within their own application for analysis.  This can
be done using the ProfilerApp::load command.  This is a static member function
that will load all of the timer data for a set of results or a specific rank.  
Its interface is:
\begin{Verbatim}[fontsize=\normal,baselinestretch=1.0]
    TimerMemoryResults  ProfilerApp::load(filename,rank);
\end{Verbatim}
The value for rank is optional (default value is -1), and a value of -1 indicates to
load all results.  
The TimerMemoryResults structure is defined in ProfilerApp.h.  


%%%---------------------------------------------------------------------------%%
%%% Bibliography
%%%---------------------------------------------------------------------------%%
%\begin{thebibliography}{10}

%\bibitem{Kuba03}
%Jaroslav Kuba, Djamel Benredjem, Clary Mo¨ller, Ladislav Drsˇka
%\newblock {\em "Analytical and numerical ray tracing of a transient x-ray laser:
%Ni-like Ag laser at 13.9 nm"},
%\newblock Journal of the Optical Society of America B, Vol. 20, No. 3, (2003).

%\end{thebibliography}


\end{document}



